\documentclass[a4paper,11pt]{article}

\usepackage[T1]{polski}
\usepackage[utf8]{inputenc} 
\usepackage{graphicx}
\usepackage{float}
\usepackage{verbatim}
\hoffset=-3.0cm                 % Mniejszy lewy margines
\textwidth=18cm                 % szerzej
\evensidemargin=0pt

\voffset=-3cm                   % Mniejszy gorny margines
\textheight=27cm                % szerzej wzdluz

\usepackage{listings}
% Listingi
\lstdefinestyle{customc}{
	belowcaptionskip=1\baselineskip,
	breaklines=true,
	frame=L,
	xleftmargin=0pt,
	language=HTML,
	showstringspaces=false,
	basicstyle=\footnotesize\ttfamily,
	identifierstyle=\color{black}
}

\setlength{\parindent}{0pt}             % No paragraph indentation
\setlength{\parskip}{\medskipamount}    % Space between paragraphs
\raggedbottom   


\title{POLITECHNIKA WARSZAWSKA \\ WYDZIAŁ ELEKTRYCZNY \\}
\author{Karol Mąkosa}
\date{\today}

\begin{document}
	\thispagestyle{empty}
	\maketitle
	\date{}
	\section{Treść zadania}
	Napisać program rozwiązujący problem komiwojażera (minimalizacja drogi pomiędzy n miastami bez powtórzeń) przy pomocy algorytmu genetycznego. Zastosować reprodukcję przy użyciu nieproporcjonalnej ruletki, operator krzyżowania PMX, oraz mutację równomierną.\\~\\
	Program powinien umożliwiać użycie różnych wielkości populacji, liczby iteracji, prawdopodobieństwa mutacji.\\~\\
	Program powinien zapewnić wizualizację wyników w postaci wykresów średniego, maksymalnego i minimalnego przystosowania (długości trasy) dla kolejnych populacji oraz 2 map (o wymiarach 10x10 punktów), na których będą wyświetlane miasta oraz drogi nadłuższa i najkrótsza.\\~\\
	Pokazać działanie programu na danych testowych składająch się z 10 miast, opisanych za pomocą współrzędnyc na mapie o wymiarach 10x10 punktów.\\~\\
	Dane testowe:miasta:\\
	A(4, 4), B(1, 10), C(8, 8), D(3, 10), E(5, 9), F(7, 8), G(6, 5), H(2, 7), I(9, 6), J(10, 3)
	\section{Instrukcja działania programu}
		\subsection{Okno główne programu}
			\begin{figure}[H]
				\centering
				\includegraphics[scale=0.5]{okno.jpg}
			\end{figure}
		\newpage
		\subsection{Opis okna programu}
				\begin{enumerate}
					\item Sterowanie programem -- ustawianie prawdopodobieństwa mutacji, liczebności populacji, liczby iteracji
					\item Mapy pokazujące najlepszą i najgorszą trasę obecnej populacji
					\item Wykresy: minimalnego, średniego i maksymalnego przystosowania oraz informacja o liczbie przebytych iteracjach
					\item Mapa pokazująca globalnie najkrótszą trasę
				\end{enumerate}	
		\subsection{Zmiana ustawień programu}
			W programie okienkowym można zmieniać liczebność populacji, prawodpodobieństwa mutacji oraz liczbę iteracji do wykonania.\\
			Aby zmienić współrzędne miast, należy zmodyfikować pole \textit{self.data} w początkowych liniach pliku \textit{Population.py}
		\subsection{Przykładowy sposób użycia}
		Aby poprawnie użyć programu należy:
		\begin{enumerate}
		\item Ustawić dane -- prawdopodobieństwo mutacji, liczebność populacji
		\item Zaakcpetować ustawienia wciskając przycisk -- \texttt{Zastosuj}
		\item Wybrać liczbę iteracji, którą chcemy wykonać
		\item Zacząć generować kolejne interacje populacji wciskając przycisk -- \texttt{Start}
		\end{enumerate}
		
	\section{Opis eksperymentów}
	Eksperymenty przeprowadzone były dla różnych wartości prawdopodobieństwa mutacji, liczebności populacji oraz iteracji
	
		\subsection{Różna liczebność populacji}
			Eksperymenty badające wpływ liczebności na wyniki algorytmu genetycznego wykonywane były przy \textbf{prawdopodobieństwie mutacji równym 0,02}. W tabeli umieszczone zostały: numer iteracji, minimalne, maksymalne i średnie przystosowanie danej populacji.
			\subsubsection{Populacja: 10 osobników}
				\begin{tabular}{|c|c|c|c|c|}
					\hline 
					Iteracja &  Min &  Max & Avg & Best\\
					\hline
					10 & 31.95 & 46.68 & 38.66 & 30.83\\
					\hline
					20 & 32.8 & 47.66 & 36.8 & 29.53\\
					\hline
					30 & 29.9 & 52.63 & 35.32 & 27.77\\
					\hline
					40 & 29.9 & 39.87 & 32.53 & 27.77\\
					\hline
					50 & 27.96 & 49.98 & 37.45 & 27.77\\
					\hline
					60 & 32.74 & 47.18 & 36.72 & 27.77\\
					\hline
					70 & 27.03 & 54.42 & 39.5 & 27.03\\
					\hline
					80 & 27.03 & 43.75 & 33.11 & 24.35\\
					\hline
					90 & 27.03 & 43.31 & 32.04 & 24.35\\
					\hline
					100 & 27.84 & 50.0 & 37.58 & 24.35\\
					\hline
				\end{tabular} \\
			\subsubsection{Populacja: 20 osobników}
				\begin{tabular}{|c|c|c|c|c|}
					\hline 
					Iteracja &  Min &  Max & Avg & Best\\
					\hline
					10 & 34.98 & 54.5 & 43.81 & 30.01\\
					\hline
					20 & 32.74 & 53.27 & 43.18 & 30.01\\
					\hline
					30 & 31.45 & 46.94 & 36.84 & 30.01\\
					\hline
					40 & 30.52 & 49.3 & 36.8 & 27.45\\
					\hline
					50 & 26.87 & 47.27 & 36.29 & 25.35\\
					\hline
					60 & 28.04 & 48.67 & 37.55 & 24.48\\
					\hline
					70 & 28.94 & 43.52 & 36.66 & 24.48\\
					\hline
					80 & 29.58 & 47.61 & 33.73 & 24.48\\
					\hline
					90 & 27.19 & 33.88 & 31.35 & 24.48\\
					\hline
					100 & 25.56 & 46.78 & 32.92 & 24.48\\
					\hline
				\end{tabular} \\
			\subsubsection{Populacja: 50 osobników}
				\begin{tabular}{|c|c|c|c|c|}
					\hline 
					Iteracja &  Min &  Max & Avg & Best\\
					\hline
					10 & 25.07 & 53.25 & 43.26 & 25.07\\
					\hline
					20 & 29.38 & 51.3 & 41.94 & 25.07\\
					\hline
					30 & 27.81 & 57.26 & 41.77 & 25.07\\
					\hline
					40 & 26.01 & 49.18 & 39.66 & 25.07\\
					\hline
					50 & 25.04 & 55.16 & 37.94 & 23.73\\
					\hline
					60 & 23.73 & 50.54 & 35.56 & 22.8\\
					\hline
					70 & 27.78 & 50.7 & 37.44 & 22.8\\
					\hline
					80 & 23.73 & 47.83 & 36.17 & 22.8\\
					\hline
					90 & 29.04 & 49.14 & 37.14 & 22.8\\
					\hline
					100 & 23.73 & 51.29 & 37.1 & 22.8\\
					\hline
				\end{tabular} \\
			\subsubsection{Populacja: 100 osobników}
				\begin{tabular}{|c|c|c|c|c|}
					\hline 
					Iteracja &  Min &  Max & Avg & Best\\
					\hline
					10 & 33.49 & 56.43 & 43.57 & 30.44\\
					\hline
					20 & 33.46 & 55.91 & 42.64 & 27.36\\
					\hline
					30 & 30.03 & 53.29 & 42.47 & 26.17\\
					\hline
					40 & 33.8 & 56.74 & 44.23 & 26.17\\
					\hline
					50 & 32.89 & 59.43 & 44.39 & 26.17\\
					\hline
					60 & 30.29 & 57.85 & 44.06 & 26.17\\
					\hline
					70 & 31.92 & 54.75 & 44.1 & 26.17\\
					\hline
					80 & 30.14 & 53.92 & 43.07 & 24.65\\
					\hline
					90 & 28.03 & 58.81 & 42.87 & 24.65\\
					\hline
					100 & 29.1 & 54.26 & 42.45 & 24.65\\
					\hline
				\end{tabular} \\
		\subsection{Różne prawdopodobieństwa mutacji}
			Eksperymenty przeprowadzone będą dla \textbf{populacji 20 osobników}.
			\subsubsection{Prawdopodobieństwo mutacji: 0,01}
				\begin{tabular}{|c|c|c|c|c|}
					\hline 
					Iteracja &  Min &  Max & Avg & Best\\
					\hline
					10 & 36.59 & 56.84 & 45.11 & 31.5\\
					\hline
					20 & 36.2 & 50.57 & 42.29 & 31.5\\
					\hline
					30 & 30.41 & 39.12 & 34.03 & 26.83\\
					\hline
					40 & 30.41 & 49.4 & 36.85 & 26.83\\
					\hline
					50 & 28.81 & 43.25 & 33.75 & 26.83\\
					\hline
					60 & 28.81 & 44.54 & 33.19 & 26.83\\
					\hline
					70 & 25.56 & 34.45 & 29.12 & 25.56\\
					\hline
					80 & 25.56 & 40.86 & 28.26 & 25.56\\
					\hline
					90 & 25.56 & 34.18 & 27.73 & 25.56\\
					\hline
					100 & 25.56 & 40.85 & 29.28 & 25.56\\
					\hline
				\end{tabular} \\
			\subsubsection{Prawdopodobieństwo mutacji: 0,03}
				\begin{tabular}{|c|c|c|c|c|}
					\hline 
					Iteracja &  Min &  Max & Avg & Best\\
					\hline
					10 & 28.14 & 49.51 & 36.41 & 28.14\\
					\hline
					20 & 36.1 & 52.67 & 43.66 & 28.14\\
					\hline
					30 & 33.17 & 53.1 & 41.04 & 28.14\\
					\hline
					40 & 31.01 & 50.55 & 37.77 & 28.14\\
					\hline
					50 & 30.99 & 53.07 & 41.26 & 28.14\\
					\hline
					60 & 32.38 & 49.59 & 39.63 & 28.14\\
					\hline
					70 & 32.68 & 53.2 & 41.72 & 28.14\\
					\hline
					80 & 29.63 & 49.01 & 40.85 & 28.14\\
					\hline
					90 & 28.76 & 45.18 & 34.56 & 25.47\\
					\hline
					100 & 29.45 & 41.74 & 33.08 & 25.47\\
					\hline
				\end{tabular} \\
			\subsubsection{Prawdopodobieństwo mutacji: 0,07}
				\begin{tabular}{|c|c|c|c|c|}
					\hline 
					Iteracja &  Min &  Max & Avg & Best\\
					\hline
					10 & 35.27 & 52.64 & 41.11 & 27.52\\
					\hline
					20 & 33.8 & 53.61 & 43.94 & 27.52\\
					\hline
					30 & 33.26 & 53.07 & 44.67 & 27.52\\
					\hline
					40 & 35.88 & 51.89 & 42.87 & 27.52\\
					\hline
					50 & 36.0 & 56.16 & 44.96 & 27.52\\
					\hline
					60 & 34.22 & 54.61 & 43.06 & 27.52\\
					\hline
					70 & 36.08 & 52.64 & 43.62 & 27.52\\
					\hline
					80 & 34.43 & 51.92 & 43.55 & 27.52\\
					\hline
					90 & 31.37 & 51.34 & 42.58 & 27.52\\
					\hline
					100 & 36.53 & 55.38 & 46.07 & 27.52\\
					\hline
				\end{tabular} \\
			\subsubsection{Prawdopodobieństwo mutacji: 0,13}
			\begin{tabular}{|c|c|c|c|c|}
				\hline 
				Iteracja &  Min &  Max & Avg & Best\\
				\hline
				10 & 30.1 & 50.61 & 41.41 & 30.1\\
				\hline
				20 & 34.46 & 50.61 & 42.1 & 30.1\\
				\hline
				30 & 32.55 & 57.22 & 43.1 & 29.53\\
				\hline
				40 & 39.11 & 53.86 & 45.02 & 29.53\\
				\hline
				50 & 36.1 & 47.13 & 41.43 & 29.53\\
				\hline
				60 & 37.16 & 51.73 & 45.36 & 29.53\\
				\hline
				70 & 35.69 & 52.22 & 43.44 & 28.23\\
				\hline
				80 & 36.23 & 50.19 & 42.81 & 28.23\\
				\hline
				90 & 37.03 & 49.22 & 43.13 & 28.23\\
				\hline
				100 & 35.9 & 54.55 & 45.24 & 28.23\\
				\hline
			\end{tabular} \\
	\section{Wnioski}
	Na zróżnicowanie wyników algorytmu genetycznego mają wpływ:
	\begin{itemize}
		\item Liczebność populacji -- ma ona wpływ na szybkość odnajdowania najkrótszych tras. Pomaga ona w odnajdowaniu najkrótszej trasy globalnie, jednakże nie wpływa na średnią odnalezioną trasę w tak dużym stopniu.
		\item Prawdopodobieństwo mutacji -- wprowadza losowe zmiany między populacjami. Nie może być ona zbyt duża, gdyż wtedy algorytm działałby w sposób pseudolosowy.
		W związku z tym, że zmiany są losowe ciężko przewidzieć czy mutacja poprawi wyniki algorymtu czy też je pogorszy. W przypadku zbyt dużej mutacji -- większej niż 3\% widać pogorszenie wyników dawanych przez program. Nagłe skoki w kolejnych iteracjach są powodowane przez niekorzystną mutację najlepszych wyników.
		\item Liczba iteracji -- Główny czynnik zmniejszania średniej trasy. Im więcej iteracji wykonamy tym lepsze otrzymujemy wyniki.
	\end{itemize}
	
	
	
\end{document}